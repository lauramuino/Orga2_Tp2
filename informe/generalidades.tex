\subsection{Comentarios Generales}
	
	El principal objetivo de este trabajo pr\'actico es ver como utilizar las instrucciones $SSE$ disponibles en un procesador x86-64 para procesar datos provenientes de im\'agenes de una forma paralela y con pocos ciclos de clock y ver como se compara con la versi\'on en $C$ que generalmente no aprovecha estos registros y utiliza mucho m\'as la memoria.
	
	La administraci\'on de frames de video como im\'agenes y la apertura o clausura de archivos de video no fueron la principal preocupaci\'on del trabajo ya que se nos provey\'o de antemano un framework para poder procesar video y de esta manera poder enfocarse solamente en el desarrollo de los filtros. 
	
	Una cosa que se hizo evidente mientras se desarrollaba el trabajo es como al estar en control directamente de las instrucciones del procesador y sin tener que pasar por un compilador se puede obtener implementaciones m\'as r\'apidas en $ASM$ que en $C$. Sin embargo, programar en $ASM$, un lenguaje ensamblador, pone mucha m\'as carga sobre el programador a la hora de tratar los datos como corresponde; esto no sucede en $C$, un lenguaje fuertemente tipado donde todas las variables tienen un tipo definido y m\'etodos asociados a ellos. Uno tiene que ser m\'as responsable cuando lee datos en registros o memoria ya que no hay forma de saber autom\'aticamente con que tipo de datos estamos operando.
