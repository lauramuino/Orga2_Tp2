El trabajo pr\'actico realizado consiti\'o en desarrollar diferentes filtros de video y un procesador de mensajes ocultos en im\'agenes. Para analizar las diferentes opciones que se ten\'ian a la hora de hacer los filtros se hicieron en principio dos versiones: una en lenguaje $C$ y otra en lenguaje ensamblador($ASM$). Una vez realizadas esas implementaciones se procedi\'o a comparar estructuralmente el c\'odigo generado por el compilador de $C$ con el c\'odigo fuente de $ASM$, observando que decisiones tomaba el compilador. Luego, para determinados filtros implementados en $C$, se empez\'o a utilizar diferentes optimizaciones a la hora de compilar con el objetivo de obtener c\'odigo con tiempo de ejecuci\'on menor al original. Tambi\'en se analizaron diferentes t\'ecnicas de optimizaci\'on de c\'odigo $ASM$ para intentar mejorar el tiempo de ejecuci\'on del mismo. Con estas diferentes alternativas se realizaron mediciones de tiempo para as\'i realizar una comparaci\'on precisa de las diferencias en tiempo de ejecuci\'on y viendo si las optimizaciones de compilaci\'on efectivamente lograron una mejor\'ia en tiempo de ejecuci\'on.


